\documentclass[3pt,a5paper,twoside]{article} % 10pt font size, A4 paper and two-sided margins

\usepackage[top=1.3cm,bottom=1.3cm,left=1.3cm,right=1.3cm,columnsep=20pt]{geometry} % Document margins and spacings

\usepackage[utf8]{inputenc} % Required for inputting international characters
\usepackage[T1]{fontenc} % Output font encoding for international characters
\usepackage{fontspec} % Output font encoding for international characters

\usepackage{microtype} % Improves spacing
\usepackage{xeCJK}
\usepackage{xpinyin}
\usepackage{qrcode}
\usepackage{tipa}
\usepackage{multicol} % Required for splitting text into multiple columns

\usepackage[bf,sf,center]{titlesec} % Required for modifying section titles - bold, sans-serif, centered

\usepackage{fancyhdr} % Required for modifying headers and footers
\usepackage{lettrine}

\setCJKmainfont{Noto Sans CJK HK}
\setmainfont{DejaVu Serif}
\fancyhead[L]{\textsf{\rightmark}} % Top left header
\fancyhead[R]{\textsf{\leftmark}} % Top right header
\renewcommand{\headrulewidth}{1.4pt} % Rule under the header
\fancyfoot[C]{\textbf{\textsf{\thepage}}} % Bottom center footer
\renewcommand{\footrulewidth}{1.4pt} % Rule under the footer
\pagestyle{fancy} % Use the custom headers and footers throughout the document

\newcommand{\jk}[1]{\textbf{#1}}
\newcommand{\entry}[2]{\textbf{#1} - #2}
\newcommand{\centry}[2]{#1 - #2}
%\newcommand{\entry}[4]{\markboth{#1}{#1}\textbf{#1}\ {(#2)}\ \textit{#3}\ $\bullet$\ {#4}}  % Defines the command to print each word on the page, \markboth{}{} prints the first word on the page in the top left header and the last word in the top right


\begin{document}

\lettrine{A}{s} we approach the final weeks of the Lenten period, the joy that should accompany our expectation of the Lord's Resurrection is often coupled with a sense of disappointment.
The zeal with which we approached fasting, heightened prayer, spiritual reading and charity in the first week of Lent has for many been replaced by a feeling of despondency. Perhaps we have failed to keep the fast, been slothful in prayer or neglected the poor. Or, perhaps, despite keeping all the ``rules'' to the letter, we find that we have not really changed, grown closer to God or become better people as a result.
Perhaps they even became an occasion for pride and self-aggrandizement, rather than the humility and love of God and others they are meant to produce. St. John of Sinai, whose memory we keep today, sums this up beautifully in his book The Ladder of Divine Ascent, undoubtedly one of the most important works on Christian spirituality ever written:

\begin{center}
    \textit{``I am vainglorious when I fast; and when I relax the fast in order to be unnoticed, I am again vainglorious over my prudence. When well-dressed I am quite overcome by vainglory and when I put on my poor clothes I am vainglorious again. When I talk I am defeated and when I am silent I am again defeated by it. However I throw this prickly-pear, a spike stands upright'' (Step 22, 5)}
\end{center}

    Aware of our struggle, the Church gives us hope and encouragement through the words of the author of the Epistle to the Hebrews. In the midst of seeming hopelessness and failure, yesterday's reading promises us that ``God is not so unjust as to overlook your work and the love which you showed for His sake'', while today's reading continues with the example of Abraham who, ``having patiently endured, obtained the promise... We have this as a sure hope and steadfast anchor of the soul''. The father in today's Gospel reading was also aware of his shortcomings and failures, which is why he did not just say ``I believe'' but added, ``help my unbelief''. Yet this modest plea of help was enough to free his son from the tyranny of demons!
    Recognising our weaknesses and failings should not be a reason to despair or give up. Quite the contrary, even if it is all we achieve this Lent, such a realisation is the first step towards genuine humility and trust in god, without which we cannot begin to climb the ladder of divine ascent. As the same St. John says elsewhere, ``Let your prayers be simple, for both the Publican and the Prodigal Son were reconciled to God by a single phrase''. Let us therefore draw strength from this short petition - Lord, I believe, help my unbelief! - as we renew our efforts during these final weeks of the fast, remembering the words of our Saviour: ``This kind cannot be driven out by anything but prayer and fasting''.


\begin{center}
    \textit{``Our relentless enemy, the teacher of fornication, whispers that God is lenient and particularly merciful to this passion, since it is so very natural. Yet if we watch the wiles of the demons we will observe that after we have actually sinned they will affirm that God is a just and inexorable judge. They say one thing to lead us into sin, another thing to overwhelm us in despair. And if we are sorrowful or inclined to despair, we are slower to sin again but when the sorrow and the despair have been quenched, the tyrannical demon begins to speak to us again of God's mercy'' (St. John of the Ladder)}
\end{center}
\end{document}
